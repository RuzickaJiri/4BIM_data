\documentclass{article}

\usepackage{Sweave}
\begin{document}
\input{TP-concordance}

\section{Blé}
\begin{Schunk}
\begin{Sinput}
> setwd("~/4A/Stats")
> ble = read.table("ble.txt", header = T, dec = ".")
> head(ble)
\end{Sinput}
\begin{Soutput}
  parcelle variete rendement
1        1       A     61.00
2        2       A     70.92
3        3       A     55.94
4        1       L     70.80
5        2       L     66.44
6        3       L     68.12
\end{Soutput}
\begin{Sinput}
> attach(ble)
> parcelle_f = as.factor(parcelle)
> table(parcelle_f,variete)
\end{Sinput}
\begin{Soutput}
          variete
parcelle_f A L M
         1 1 1 1
         2 1 1 1
         3 1 1 1
\end{Soutput}
\end{Schunk}
Croisé - toutes les modalités sont analysées à travers toutes les modalités
Complet - 
Contraire de croisé est hierarchisé - 

\begin{Schunk}
\begin{Sinput}
> lm1 = lm(rendement~parcelle_f*variete)
> anova(lm1)
\end{Sinput}
\begin{Soutput}
Analysis of Variance Table

Response: rendement
                   Df Sum Sq Mean Sq F value Pr(>F)
parcelle_f          2  43.86  21.930               
variete             2 246.01 123.004               
parcelle_f:variete  4  86.38  21.595               
Residuals           0   0.00                       
\end{Soutput}
\end{Schunk}

\begin{Schunk}
\begin{Sinput}
> lm2 = lm(rendement~parcelle_f+variete)
> anova(lm2)
\end{Sinput}
\begin{Soutput}
Analysis of Variance Table

Response: rendement
           Df Sum Sq Mean Sq F value  Pr(>F)  
parcelle_f  2  43.86  21.930  1.0155 0.43988  
variete     2 246.01 123.004  5.6959 0.06754 .
Residuals   4  86.38  21.595                  
---
Signif. codes:  0 ‘***’ 0.001 ‘**’ 0.01 ‘*’ 0.05 ‘.’ 0.1 ‘ ’ 1
\end{Soutput}
\end{Schunk}
Si les variétés étaient aléatoires, parcelles aussi - modèle complètement aléatoire
On doit tester par rapport à l'interaction
\begin{Schunk}
\begin{Sinput}
> interaction.plot(parcelle_f,variete,rendement)
> lm3 = lm(rendement~variete)
> anova(lm3)
\end{Sinput}
\begin{Soutput}
Analysis of Variance Table

Response: rendement
          Df Sum Sq Mean Sq F value  Pr(>F)  
variete    2 246.01 123.004  5.6666 0.04148 *
Residuals  6 130.24  21.707                  
---
Signif. codes:  0 ‘***’ 0.001 ‘**’ 0.01 ‘*’ 0.05 ‘.’ 0.1 ‘ ’ 1
\end{Soutput}
\end{Schunk}

lm3 - modèle à un facteur
on teste variété à l'intéraction + parcelle - c'est la triche, on néglige les parcelles
cela nous fait perdre une information dans nos données qui est fondamentale

\section{Prodlait}
\begin{Schunk}
\begin{Sinput}
> prod = read.table("prodlait.dat", header = T, dec = ".")
> head(prod)
\end{Sinput}
\begin{Soutput}
  Regime  race prodlait
1   Haut race1       33
2   Haut race1       35
3   Haut race1       36
4   Haut race1       43
5    Bas race1       31
6    Bas race1       32
\end{Soutput}
\begin{Sinput}
> attach(prod)
> interaction.plot(race,Regime,prodlait)
> lm1 = lm(prodlait~race*Regime)
> anova(lm1)
\end{Sinput}
\begin{Soutput}
Analysis of Variance Table

Response: prodlait
            Df Sum Sq Mean Sq F value    Pr(>F)    
race         1  181.5  181.50   29.04 4.033e-05 ***
Regime       2   49.0   24.50    3.92   0.03862 *  
race:Regime  2    3.0    1.50    0.24   0.78911    
Residuals   18  112.5    6.25                      
---
Signif. codes:  0 ‘***’ 0.001 ‘**’ 0.01 ‘*’ 0.05 ‘.’ 0.1 ‘ ’ 1
\end{Soutput}
\end{Schunk}
Race - effet aléatoire - très significatif
Regime - effet fixe - on a pas le droit de regarder


\begin{Schunk}
\begin{Sinput}
> lm2 = lm(prodlait~Regime*race)
> anova(lm2)
\end{Sinput}
\begin{Soutput}
Analysis of Variance Table

Response: prodlait
            Df Sum Sq Mean Sq F value    Pr(>F)    
Regime       2   49.0   24.50    3.92   0.03862 *  
race         1  181.5  181.50   29.04 4.033e-05 ***
Regime:race  2    3.0    1.50    0.24   0.78911    
Residuals   18  112.5    6.25                      
---
Signif. codes:  0 ‘***’ 0.001 ‘**’ 0.01 ‘*’ 0.05 ‘.’ 0.1 ‘ ’ 1
\end{Soutput}
\end{Schunk}
Pas intéressant

\begin{Schunk}
\begin{Sinput}
> lm2 = aov(prodlait~Regime+race+Error(Regime:race))
> summary(lm3)
\end{Sinput}
\begin{Soutput}
Call:
lm(formula = rendement ~ variete)

Residuals:
    Min      1Q  Median      3Q     Max 
-6.6800 -1.6200 -0.3633  1.6367  8.3000 

Coefficients:
            Estimate Std. Error t value Pr(>|t|)    
(Intercept)   62.620      2.690  23.280 4.12e-07 ***
varieteL       5.833      3.804   1.533    0.176    
varieteM      -6.957      3.804  -1.829    0.117    
---
Signif. codes:  0 ‘***’ 0.001 ‘**’ 0.01 ‘*’ 0.05 ‘.’ 0.1 ‘ ’ 1

Residual standard error: 4.659 on 6 degrees of freedom
Multiple R-squared:  0.6538,	Adjusted R-squared:  0.5385 
F-statistic: 5.667 on 2 and 6 DF,  p-value: 0.04148
\end{Soutput}
\end{Schunk}
Pour faire correctement un modèle mixte avec répetition
2 modèles : 1 lm et 1 aov
la race - l'effet aléatoire - on ne peut pas intérpreter la race sur le modèle aov
regime - effet fixe - pas significatif, très limite

La race introduit un effet aléatoire supplementaire dans l'exercice.

Estimer les composantes de covariance - estimer sigma interaction et sigma race

(1.5 - 6.25)/4 = 0
E(Vsab) - variance/n


\section{Tournesol}
\begin{Schunk}
\begin{Sinput}
> setwd("~/4A/Stats")
> tournesol = read.table("tournesol.txt", header = T, dec = ".")
> head(tournesol)
\end{Sinput}
\begin{Soutput}
  teneur testeur origine
1  43.54      T1 afrique
2  45.30      T1 afrique
3  44.65      T1 afrique
4  47.21      T2 afrique
5  47.73      T2 afrique
6  47.23      T2 afrique
\end{Soutput}
\begin{Sinput}
> attach(tournesol)
\end{Sinput}
\end{Schunk}
plan factoriel à effet mixte
on a à la fois effet aléatoire et effet melangé
plan croisé complet équilibré

Hypothèses :  independance des variables - difficile à tester - dans l'énoncé
              normalité des résidues - à posteriori
              variance constante - ne depend pas de lignes et de colonnes
              
bartlett - tester les variances
\begin{Schunk}
\begin{Sinput}
> bartlett.test(teneur, testeur:origine)
\end{Sinput}
\begin{Soutput}
	Bartlett test of homogeneity of variances

data:  teneur and testeur:origine
Bartlett's K-squared = 3.3366, df = 5, p-value = 0.6482
\end{Soutput}
\begin{Sinput}
> testeur:origine
\end{Sinput}
\begin{Soutput}
 [1] T1:afrique T1:afrique T1:afrique T2:afrique T2:afrique T2:afrique
 [7] T1:hongrie T1:hongrie T1:hongrie T2:hongrie T2:hongrie T2:hongrie
[13] T1:maroc   T1:maroc   T1:maroc   T2:maroc   T2:maroc   T2:maroc  
Levels: T1:afrique T1:hongrie T1:maroc T2:afrique T2:hongrie T2:maroc
\end{Soutput}
\end{Schunk}
homocédasticité - égalité des variances

\begin{Schunk}
\begin{Sinput}
> lm1 = lm(teneur~testeur*origine)
\end{Sinput}
\end{Schunk}
plan est équilibré et donc les décompositions carrées sont orthogonales
testeur*origine ~ origine*testeur

\begin{Schunk}
\begin{Sinput}
> summary(lm1)
\end{Sinput}
\begin{Soutput}
Call:
lm(formula = teneur ~ testeur * origine)

Residuals:
     Min       1Q   Median       3Q      Max 
-1.40667 -0.76917 -0.00167  0.66250  1.56333 

Coefficients:
                         Estimate Std. Error t value Pr(>|t|)    
(Intercept)               44.4967     0.6057  73.465  < 2e-16 ***
testeurT2                  2.8933     0.8566   3.378 0.005490 ** 
originehongrie            -0.7367     0.8566  -0.860 0.406633    
originemaroc               3.7400     0.8566   4.366 0.000918 ***
testeurT2:originehongrie  -1.0767     1.2114  -0.889 0.391581    
testeurT2:originemaroc    -3.2233     1.2114  -2.661 0.020757 *  
---
Signif. codes:  0 ‘***’ 0.001 ‘**’ 0.01 ‘*’ 0.05 ‘.’ 0.1 ‘ ’ 1

Residual standard error: 1.049 on 12 degrees of freedom
Multiple R-squared:  0.801,	Adjusted R-squared:  0.718 
F-statistic: 9.658 on 5 and 12 DF,  p-value: 0.0006888
\end{Soutput}
\begin{Sinput}
> anova(lm1)
\end{Sinput}
\begin{Soutput}
Analysis of Variance Table

Response: teneur
                Df Sum Sq Mean Sq F value    Pr(>F)    
testeur          1  9.592  9.5922  8.7156 0.0120941 *  
origine          2 35.476 17.7381 16.1172 0.0003986 ***
testeur:origine  2  8.079  4.0393  3.6702 0.0570553 .  
Residuals       12 13.207  1.1006                      
---
Signif. codes:  0 ‘***’ 0.001 ‘**’ 0.01 ‘*’ 0.05 ‘.’ 0.1 ‘ ’ 1
\end{Soutput}
\end{Schunk}
anova - On peut regarder que le testeur - effet aléatoire

\begin{Schunk}
\begin{Sinput}
> interaction.plot(testeur, origine, teneur)
\end{Sinput}
\end{Schunk}

\begin{Schunk}
\begin{Sinput}
> lm2 = aov(teneur~origine+testeur+Error(testeur:origine))
> summary(lm2)
\end{Sinput}
\begin{Soutput}
Error: testeur:origine
          Df Sum Sq Mean Sq F value Pr(>F)
origine    2  35.48  17.738   4.391  0.185
testeur    1   9.59   9.592   2.375  0.263
Residuals  2   8.08   4.039               

Error: Within
          Df Sum Sq Mean Sq F value Pr(>F)
Residuals 12  13.21   1.101               
\end{Soutput}
\end{Schunk}
les machines apportent un biais dans l'analyse
les huiles sont les memes wesh

\begin{Schunk}
\begin{Sinput}
> par(mfrow=c(2,2))
> plot(lm1)
\end{Sinput}
\end{Schunk}

\begin{Schunk}
\begin{Sinput}
> shapiro.test(lm1$residuals)
\end{Sinput}
\begin{Soutput}
	Shapiro-Wilk normality test

data:  lm1$residuals
W = 0.95552, p-value = 0.518
\end{Soutput}
\end{Schunk}

Test de Levene
\begin{Schunk}
\begin{Sinput}
> lm3 = lm(abs(lm1$residuals)~origine*testeur)
> anova(lm3)
\end{Sinput}
\begin{Soutput}
Analysis of Variance Table

Response: abs(lm1$residuals)
                Df  Sum Sq Mean Sq F value Pr(>F)
origine          2 0.76573 0.38287  1.6538 0.2321
testeur          1 0.00802 0.00802  0.0347 0.8554
origine:testeur  2 0.35264 0.17632  0.7616 0.4882
Residuals       12 2.77805 0.23150               
\end{Soutput}
\end{Schunk}
teste la linéarité du modèle
test global 


\section{Psouris}

\begin{Schunk}
\begin{Sinput}
> setwd("~/4A/Stats")
> pdsouris = read.table("pdsouris.txt", header = T, dec = ".")
> head(pdsouris)
\end{Sinput}
\begin{Soutput}
  pere mere poids
1    A    1  19.3
2    A    1  21.9
3    A    2  22.7
4    A    2  24.6
5    A    3  21.0
6    A    3  19.1
\end{Soutput}
\begin{Sinput}
> #attach(pdsouris)
> mere=as.factor(pdsouris$mere)
> pere = pdsouris$pere
\end{Sinput}
\end{Schunk}
c'est pas un plan croisé
plan complet - tous les croisements sont faits
plan croisé - on peut croiser toutes les modalités de 1 avec toutes les modalités de l'autre

\begin{Schunk}
\begin{Sinput}
> lm1 = lm(poids~pere+mere)