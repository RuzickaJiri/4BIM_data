
% Default to the notebook output style

    


% Inherit from the specified cell style.




    
\documentclass[11pt]{article}

    
    
    \usepackage[T1]{fontenc}
    % Nicer default font (+ math font) than Computer Modern for most use cases
    \usepackage{mathpazo}

    % Basic figure setup, for now with no caption control since it's done
    % automatically by Pandoc (which extracts ![](path) syntax from Markdown).
    \usepackage{graphicx}
    % We will generate all images so they have a width \maxwidth. This means
    % that they will get their normal width if they fit onto the page, but
    % are scaled down if they would overflow the margins.
    \makeatletter
    \def\maxwidth{\ifdim\Gin@nat@width>\linewidth\linewidth
    \else\Gin@nat@width\fi}
    \makeatother
    \let\Oldincludegraphics\includegraphics
    % Set max figure width to be 80% of text width, for now hardcoded.
    \renewcommand{\includegraphics}[1]{\Oldincludegraphics[width=.8\maxwidth]{#1}}
    % Ensure that by default, figures have no caption (until we provide a
    % proper Figure object with a Caption API and a way to capture that
    % in the conversion process - todo).
    \usepackage{caption}
    \DeclareCaptionLabelFormat{nolabel}{}
    \captionsetup{labelformat=nolabel}

    \usepackage{adjustbox} % Used to constrain images to a maximum size 
    \usepackage{xcolor} % Allow colors to be defined
    \usepackage{enumerate} % Needed for markdown enumerations to work
    \usepackage{geometry} % Used to adjust the document margins
    \usepackage{amsmath} % Equations
    \usepackage{amssymb} % Equations
    \usepackage{textcomp} % defines textquotesingle
    % Hack from http://tex.stackexchange.com/a/47451/13684:
    \AtBeginDocument{%
        \def\PYZsq{\textquotesingle}% Upright quotes in Pygmentized code
    }
    \usepackage{upquote} % Upright quotes for verbatim code
    \usepackage{eurosym} % defines \euro
    \usepackage[mathletters]{ucs} % Extended unicode (utf-8) support
    \usepackage[utf8x]{inputenc} % Allow utf-8 characters in the tex document
    \usepackage{fancyvrb} % verbatim replacement that allows latex
    \usepackage{grffile} % extends the file name processing of package graphics 
                         % to support a larger range 
    % The hyperref package gives us a pdf with properly built
    % internal navigation ('pdf bookmarks' for the table of contents,
    % internal cross-reference links, web links for URLs, etc.)
    \usepackage{hyperref}
    \usepackage{longtable} % longtable support required by pandoc >1.10
    \usepackage{booktabs}  % table support for pandoc > 1.12.2
    \usepackage[inline]{enumitem} % IRkernel/repr support (it uses the enumerate* environment)
    \usepackage[normalem]{ulem} % ulem is needed to support strikethroughs (\sout)
                                % normalem makes italics be italics, not underlines
    

    
    
    % Colors for the hyperref package
    \definecolor{urlcolor}{rgb}{0,.145,.698}
    \definecolor{linkcolor}{rgb}{.71,0.21,0.01}
    \definecolor{citecolor}{rgb}{.12,.54,.11}

    % ANSI colors
    \definecolor{ansi-black}{HTML}{3E424D}
    \definecolor{ansi-black-intense}{HTML}{282C36}
    \definecolor{ansi-red}{HTML}{E75C58}
    \definecolor{ansi-red-intense}{HTML}{B22B31}
    \definecolor{ansi-green}{HTML}{00A250}
    \definecolor{ansi-green-intense}{HTML}{007427}
    \definecolor{ansi-yellow}{HTML}{DDB62B}
    \definecolor{ansi-yellow-intense}{HTML}{B27D12}
    \definecolor{ansi-blue}{HTML}{208FFB}
    \definecolor{ansi-blue-intense}{HTML}{0065CA}
    \definecolor{ansi-magenta}{HTML}{D160C4}
    \definecolor{ansi-magenta-intense}{HTML}{A03196}
    \definecolor{ansi-cyan}{HTML}{60C6C8}
    \definecolor{ansi-cyan-intense}{HTML}{258F8F}
    \definecolor{ansi-white}{HTML}{C5C1B4}
    \definecolor{ansi-white-intense}{HTML}{A1A6B2}

    % commands and environments needed by pandoc snippets
    % extracted from the output of `pandoc -s`
    \providecommand{\tightlist}{%
      \setlength{\itemsep}{0pt}\setlength{\parskip}{0pt}}
    \DefineVerbatimEnvironment{Highlighting}{Verbatim}{commandchars=\\\{\}}
    % Add ',fontsize=\small' for more characters per line
    \newenvironment{Shaded}{}{}
    \newcommand{\KeywordTok}[1]{\textcolor[rgb]{0.00,0.44,0.13}{\textbf{{#1}}}}
    \newcommand{\DataTypeTok}[1]{\textcolor[rgb]{0.56,0.13,0.00}{{#1}}}
    \newcommand{\DecValTok}[1]{\textcolor[rgb]{0.25,0.63,0.44}{{#1}}}
    \newcommand{\BaseNTok}[1]{\textcolor[rgb]{0.25,0.63,0.44}{{#1}}}
    \newcommand{\FloatTok}[1]{\textcolor[rgb]{0.25,0.63,0.44}{{#1}}}
    \newcommand{\CharTok}[1]{\textcolor[rgb]{0.25,0.44,0.63}{{#1}}}
    \newcommand{\StringTok}[1]{\textcolor[rgb]{0.25,0.44,0.63}{{#1}}}
    \newcommand{\CommentTok}[1]{\textcolor[rgb]{0.38,0.63,0.69}{\textit{{#1}}}}
    \newcommand{\OtherTok}[1]{\textcolor[rgb]{0.00,0.44,0.13}{{#1}}}
    \newcommand{\AlertTok}[1]{\textcolor[rgb]{1.00,0.00,0.00}{\textbf{{#1}}}}
    \newcommand{\FunctionTok}[1]{\textcolor[rgb]{0.02,0.16,0.49}{{#1}}}
    \newcommand{\RegionMarkerTok}[1]{{#1}}
    \newcommand{\ErrorTok}[1]{\textcolor[rgb]{1.00,0.00,0.00}{\textbf{{#1}}}}
    \newcommand{\NormalTok}[1]{{#1}}
    
    % Additional commands for more recent versions of Pandoc
    \newcommand{\ConstantTok}[1]{\textcolor[rgb]{0.53,0.00,0.00}{{#1}}}
    \newcommand{\SpecialCharTok}[1]{\textcolor[rgb]{0.25,0.44,0.63}{{#1}}}
    \newcommand{\VerbatimStringTok}[1]{\textcolor[rgb]{0.25,0.44,0.63}{{#1}}}
    \newcommand{\SpecialStringTok}[1]{\textcolor[rgb]{0.73,0.40,0.53}{{#1}}}
    \newcommand{\ImportTok}[1]{{#1}}
    \newcommand{\DocumentationTok}[1]{\textcolor[rgb]{0.73,0.13,0.13}{\textit{{#1}}}}
    \newcommand{\AnnotationTok}[1]{\textcolor[rgb]{0.38,0.63,0.69}{\textbf{\textit{{#1}}}}}
    \newcommand{\CommentVarTok}[1]{\textcolor[rgb]{0.38,0.63,0.69}{\textbf{\textit{{#1}}}}}
    \newcommand{\VariableTok}[1]{\textcolor[rgb]{0.10,0.09,0.49}{{#1}}}
    \newcommand{\ControlFlowTok}[1]{\textcolor[rgb]{0.00,0.44,0.13}{\textbf{{#1}}}}
    \newcommand{\OperatorTok}[1]{\textcolor[rgb]{0.40,0.40,0.40}{{#1}}}
    \newcommand{\BuiltInTok}[1]{{#1}}
    \newcommand{\ExtensionTok}[1]{{#1}}
    \newcommand{\PreprocessorTok}[1]{\textcolor[rgb]{0.74,0.48,0.00}{{#1}}}
    \newcommand{\AttributeTok}[1]{\textcolor[rgb]{0.49,0.56,0.16}{{#1}}}
    \newcommand{\InformationTok}[1]{\textcolor[rgb]{0.38,0.63,0.69}{\textbf{\textit{{#1}}}}}
    \newcommand{\WarningTok}[1]{\textcolor[rgb]{0.38,0.63,0.69}{\textbf{\textit{{#1}}}}}
    
    
    % Define a nice break command that doesn't care if a line doesn't already
    % exist.
    \def\br{\hspace*{\fill} \\* }
    % Math Jax compatability definitions
    \def\gt{>}
    \def\lt{<}
    % Document parameters
    \title{Optimisation\_TD3}
    
    
    

    % Pygments definitions
    
\makeatletter
\def\PY@reset{\let\PY@it=\relax \let\PY@bf=\relax%
    \let\PY@ul=\relax \let\PY@tc=\relax%
    \let\PY@bc=\relax \let\PY@ff=\relax}
\def\PY@tok#1{\csname PY@tok@#1\endcsname}
\def\PY@toks#1+{\ifx\relax#1\empty\else%
    \PY@tok{#1}\expandafter\PY@toks\fi}
\def\PY@do#1{\PY@bc{\PY@tc{\PY@ul{%
    \PY@it{\PY@bf{\PY@ff{#1}}}}}}}
\def\PY#1#2{\PY@reset\PY@toks#1+\relax+\PY@do{#2}}

\expandafter\def\csname PY@tok@w\endcsname{\def\PY@tc##1{\textcolor[rgb]{0.73,0.73,0.73}{##1}}}
\expandafter\def\csname PY@tok@c\endcsname{\let\PY@it=\textit\def\PY@tc##1{\textcolor[rgb]{0.25,0.50,0.50}{##1}}}
\expandafter\def\csname PY@tok@cp\endcsname{\def\PY@tc##1{\textcolor[rgb]{0.74,0.48,0.00}{##1}}}
\expandafter\def\csname PY@tok@k\endcsname{\let\PY@bf=\textbf\def\PY@tc##1{\textcolor[rgb]{0.00,0.50,0.00}{##1}}}
\expandafter\def\csname PY@tok@kp\endcsname{\def\PY@tc##1{\textcolor[rgb]{0.00,0.50,0.00}{##1}}}
\expandafter\def\csname PY@tok@kt\endcsname{\def\PY@tc##1{\textcolor[rgb]{0.69,0.00,0.25}{##1}}}
\expandafter\def\csname PY@tok@o\endcsname{\def\PY@tc##1{\textcolor[rgb]{0.40,0.40,0.40}{##1}}}
\expandafter\def\csname PY@tok@ow\endcsname{\let\PY@bf=\textbf\def\PY@tc##1{\textcolor[rgb]{0.67,0.13,1.00}{##1}}}
\expandafter\def\csname PY@tok@nb\endcsname{\def\PY@tc##1{\textcolor[rgb]{0.00,0.50,0.00}{##1}}}
\expandafter\def\csname PY@tok@nf\endcsname{\def\PY@tc##1{\textcolor[rgb]{0.00,0.00,1.00}{##1}}}
\expandafter\def\csname PY@tok@nc\endcsname{\let\PY@bf=\textbf\def\PY@tc##1{\textcolor[rgb]{0.00,0.00,1.00}{##1}}}
\expandafter\def\csname PY@tok@nn\endcsname{\let\PY@bf=\textbf\def\PY@tc##1{\textcolor[rgb]{0.00,0.00,1.00}{##1}}}
\expandafter\def\csname PY@tok@ne\endcsname{\let\PY@bf=\textbf\def\PY@tc##1{\textcolor[rgb]{0.82,0.25,0.23}{##1}}}
\expandafter\def\csname PY@tok@nv\endcsname{\def\PY@tc##1{\textcolor[rgb]{0.10,0.09,0.49}{##1}}}
\expandafter\def\csname PY@tok@no\endcsname{\def\PY@tc##1{\textcolor[rgb]{0.53,0.00,0.00}{##1}}}
\expandafter\def\csname PY@tok@nl\endcsname{\def\PY@tc##1{\textcolor[rgb]{0.63,0.63,0.00}{##1}}}
\expandafter\def\csname PY@tok@ni\endcsname{\let\PY@bf=\textbf\def\PY@tc##1{\textcolor[rgb]{0.60,0.60,0.60}{##1}}}
\expandafter\def\csname PY@tok@na\endcsname{\def\PY@tc##1{\textcolor[rgb]{0.49,0.56,0.16}{##1}}}
\expandafter\def\csname PY@tok@nt\endcsname{\let\PY@bf=\textbf\def\PY@tc##1{\textcolor[rgb]{0.00,0.50,0.00}{##1}}}
\expandafter\def\csname PY@tok@nd\endcsname{\def\PY@tc##1{\textcolor[rgb]{0.67,0.13,1.00}{##1}}}
\expandafter\def\csname PY@tok@s\endcsname{\def\PY@tc##1{\textcolor[rgb]{0.73,0.13,0.13}{##1}}}
\expandafter\def\csname PY@tok@sd\endcsname{\let\PY@it=\textit\def\PY@tc##1{\textcolor[rgb]{0.73,0.13,0.13}{##1}}}
\expandafter\def\csname PY@tok@si\endcsname{\let\PY@bf=\textbf\def\PY@tc##1{\textcolor[rgb]{0.73,0.40,0.53}{##1}}}
\expandafter\def\csname PY@tok@se\endcsname{\let\PY@bf=\textbf\def\PY@tc##1{\textcolor[rgb]{0.73,0.40,0.13}{##1}}}
\expandafter\def\csname PY@tok@sr\endcsname{\def\PY@tc##1{\textcolor[rgb]{0.73,0.40,0.53}{##1}}}
\expandafter\def\csname PY@tok@ss\endcsname{\def\PY@tc##1{\textcolor[rgb]{0.10,0.09,0.49}{##1}}}
\expandafter\def\csname PY@tok@sx\endcsname{\def\PY@tc##1{\textcolor[rgb]{0.00,0.50,0.00}{##1}}}
\expandafter\def\csname PY@tok@m\endcsname{\def\PY@tc##1{\textcolor[rgb]{0.40,0.40,0.40}{##1}}}
\expandafter\def\csname PY@tok@gh\endcsname{\let\PY@bf=\textbf\def\PY@tc##1{\textcolor[rgb]{0.00,0.00,0.50}{##1}}}
\expandafter\def\csname PY@tok@gu\endcsname{\let\PY@bf=\textbf\def\PY@tc##1{\textcolor[rgb]{0.50,0.00,0.50}{##1}}}
\expandafter\def\csname PY@tok@gd\endcsname{\def\PY@tc##1{\textcolor[rgb]{0.63,0.00,0.00}{##1}}}
\expandafter\def\csname PY@tok@gi\endcsname{\def\PY@tc##1{\textcolor[rgb]{0.00,0.63,0.00}{##1}}}
\expandafter\def\csname PY@tok@gr\endcsname{\def\PY@tc##1{\textcolor[rgb]{1.00,0.00,0.00}{##1}}}
\expandafter\def\csname PY@tok@ge\endcsname{\let\PY@it=\textit}
\expandafter\def\csname PY@tok@gs\endcsname{\let\PY@bf=\textbf}
\expandafter\def\csname PY@tok@gp\endcsname{\let\PY@bf=\textbf\def\PY@tc##1{\textcolor[rgb]{0.00,0.00,0.50}{##1}}}
\expandafter\def\csname PY@tok@go\endcsname{\def\PY@tc##1{\textcolor[rgb]{0.53,0.53,0.53}{##1}}}
\expandafter\def\csname PY@tok@gt\endcsname{\def\PY@tc##1{\textcolor[rgb]{0.00,0.27,0.87}{##1}}}
\expandafter\def\csname PY@tok@err\endcsname{\def\PY@bc##1{\setlength{\fboxsep}{0pt}\fcolorbox[rgb]{1.00,0.00,0.00}{1,1,1}{\strut ##1}}}
\expandafter\def\csname PY@tok@kc\endcsname{\let\PY@bf=\textbf\def\PY@tc##1{\textcolor[rgb]{0.00,0.50,0.00}{##1}}}
\expandafter\def\csname PY@tok@kd\endcsname{\let\PY@bf=\textbf\def\PY@tc##1{\textcolor[rgb]{0.00,0.50,0.00}{##1}}}
\expandafter\def\csname PY@tok@kn\endcsname{\let\PY@bf=\textbf\def\PY@tc##1{\textcolor[rgb]{0.00,0.50,0.00}{##1}}}
\expandafter\def\csname PY@tok@kr\endcsname{\let\PY@bf=\textbf\def\PY@tc##1{\textcolor[rgb]{0.00,0.50,0.00}{##1}}}
\expandafter\def\csname PY@tok@bp\endcsname{\def\PY@tc##1{\textcolor[rgb]{0.00,0.50,0.00}{##1}}}
\expandafter\def\csname PY@tok@fm\endcsname{\def\PY@tc##1{\textcolor[rgb]{0.00,0.00,1.00}{##1}}}
\expandafter\def\csname PY@tok@vc\endcsname{\def\PY@tc##1{\textcolor[rgb]{0.10,0.09,0.49}{##1}}}
\expandafter\def\csname PY@tok@vg\endcsname{\def\PY@tc##1{\textcolor[rgb]{0.10,0.09,0.49}{##1}}}
\expandafter\def\csname PY@tok@vi\endcsname{\def\PY@tc##1{\textcolor[rgb]{0.10,0.09,0.49}{##1}}}
\expandafter\def\csname PY@tok@vm\endcsname{\def\PY@tc##1{\textcolor[rgb]{0.10,0.09,0.49}{##1}}}
\expandafter\def\csname PY@tok@sa\endcsname{\def\PY@tc##1{\textcolor[rgb]{0.73,0.13,0.13}{##1}}}
\expandafter\def\csname PY@tok@sb\endcsname{\def\PY@tc##1{\textcolor[rgb]{0.73,0.13,0.13}{##1}}}
\expandafter\def\csname PY@tok@sc\endcsname{\def\PY@tc##1{\textcolor[rgb]{0.73,0.13,0.13}{##1}}}
\expandafter\def\csname PY@tok@dl\endcsname{\def\PY@tc##1{\textcolor[rgb]{0.73,0.13,0.13}{##1}}}
\expandafter\def\csname PY@tok@s2\endcsname{\def\PY@tc##1{\textcolor[rgb]{0.73,0.13,0.13}{##1}}}
\expandafter\def\csname PY@tok@sh\endcsname{\def\PY@tc##1{\textcolor[rgb]{0.73,0.13,0.13}{##1}}}
\expandafter\def\csname PY@tok@s1\endcsname{\def\PY@tc##1{\textcolor[rgb]{0.73,0.13,0.13}{##1}}}
\expandafter\def\csname PY@tok@mb\endcsname{\def\PY@tc##1{\textcolor[rgb]{0.40,0.40,0.40}{##1}}}
\expandafter\def\csname PY@tok@mf\endcsname{\def\PY@tc##1{\textcolor[rgb]{0.40,0.40,0.40}{##1}}}
\expandafter\def\csname PY@tok@mh\endcsname{\def\PY@tc##1{\textcolor[rgb]{0.40,0.40,0.40}{##1}}}
\expandafter\def\csname PY@tok@mi\endcsname{\def\PY@tc##1{\textcolor[rgb]{0.40,0.40,0.40}{##1}}}
\expandafter\def\csname PY@tok@il\endcsname{\def\PY@tc##1{\textcolor[rgb]{0.40,0.40,0.40}{##1}}}
\expandafter\def\csname PY@tok@mo\endcsname{\def\PY@tc##1{\textcolor[rgb]{0.40,0.40,0.40}{##1}}}
\expandafter\def\csname PY@tok@ch\endcsname{\let\PY@it=\textit\def\PY@tc##1{\textcolor[rgb]{0.25,0.50,0.50}{##1}}}
\expandafter\def\csname PY@tok@cm\endcsname{\let\PY@it=\textit\def\PY@tc##1{\textcolor[rgb]{0.25,0.50,0.50}{##1}}}
\expandafter\def\csname PY@tok@cpf\endcsname{\let\PY@it=\textit\def\PY@tc##1{\textcolor[rgb]{0.25,0.50,0.50}{##1}}}
\expandafter\def\csname PY@tok@c1\endcsname{\let\PY@it=\textit\def\PY@tc##1{\textcolor[rgb]{0.25,0.50,0.50}{##1}}}
\expandafter\def\csname PY@tok@cs\endcsname{\let\PY@it=\textit\def\PY@tc##1{\textcolor[rgb]{0.25,0.50,0.50}{##1}}}

\def\PYZbs{\char`\\}
\def\PYZus{\char`\_}
\def\PYZob{\char`\{}
\def\PYZcb{\char`\}}
\def\PYZca{\char`\^}
\def\PYZam{\char`\&}
\def\PYZlt{\char`\<}
\def\PYZgt{\char`\>}
\def\PYZsh{\char`\#}
\def\PYZpc{\char`\%}
\def\PYZdl{\char`\$}
\def\PYZhy{\char`\-}
\def\PYZsq{\char`\'}
\def\PYZdq{\char`\"}
\def\PYZti{\char`\~}
% for compatibility with earlier versions
\def\PYZat{@}
\def\PYZlb{[}
\def\PYZrb{]}
\makeatother


    % Exact colors from NB
    \definecolor{incolor}{rgb}{0.0, 0.0, 0.5}
    \definecolor{outcolor}{rgb}{0.545, 0.0, 0.0}



    
    % Prevent overflowing lines due to hard-to-break entities
    \sloppy 
    % Setup hyperref package
    \hypersetup{
      breaklinks=true,  % so long urls are correctly broken across lines
      colorlinks=true,
      urlcolor=urlcolor,
      linkcolor=linkcolor,
      citecolor=citecolor,
      }
    % Slightly bigger margins than the latex defaults
    
    \geometry{verbose,tmargin=1in,bmargin=1in,lmargin=1in,rmargin=1in}
    
    

    \begin{document}
    
    
    \maketitle
    
    

    
    \section{Algorithme Génétique}\label{algorithme-guxe9nuxe9tique}

    \subsection{Introduction}\label{introduction}

    Rendu : code, réponses détaillées, illustrer les propos, docstring pas
obligatoire, conclusions sur la convergence, etc.

    \subsection{Marche aléatoire}\label{marche-aluxe9atoire}

    Nous allons appliquer l'algorithme génétique à un problème assez simple.
On modélise une marche aléatoire en dimension 1 sur T pas de temps, dans
la direction d : \(x(t + 1) = x(t) + d\) avec d tiré uniformément sur le
doublet \(\{−1, 1\}\), et \(t = \{0, . . . , T − 1\}\).

On cherche à optimiser la trajectoire de la marche de sorte à ce qu'elle
reste dans un intervalle déterminé \([−r_0 ; r_0 ]\). Le paramètre
\(r_0\) sera à faire varier. Le coût d'une trajectoire sera égal au
nombre de pas situés hors de l'intervalle \([−r_0 ; r_0 ]\).

\emph{Écrire une fonction qui génère aléatoirement un génome. Elle prend
\(T\), la longueur de la marche, en paramètre d'entrée et sort un
vecteur de longueur \(T\) prenant ses valeurs dans \(\{−1, 1\}\) de
façon aléatoire.}

    \begin{Verbatim}[commandchars=\\\{\}]
{\color{incolor}In [{\color{incolor}57}]:} \PY{k+kn}{from} \PY{n+nn}{mpl\PYZus{}toolkits}\PY{n+nn}{.}\PY{n+nn}{mplot3d} \PY{k}{import} \PY{n}{Axes3D}
         \PY{k+kn}{from} \PY{n+nn}{matplotlib} \PY{k}{import} \PY{n}{cm}
         \PY{k+kn}{from} \PY{n+nn}{matplotlib}\PY{n+nn}{.}\PY{n+nn}{ticker} \PY{k}{import} \PY{n}{LinearLocator}\PY{p}{,} \PY{n}{FormatStrFormatter}
         \PY{k+kn}{import} \PY{n+nn}{matplotlib}\PY{n+nn}{.}\PY{n+nn}{pyplot} \PY{k}{as} \PY{n+nn}{plt}
         \PY{k+kn}{import} \PY{n+nn}{numpy} \PY{k}{as} \PY{n+nn}{np}
         \PY{k+kn}{import} \PY{n+nn}{random}
         \PY{k+kn}{import} \PY{n+nn}{statistics} \PY{k}{as} \PY{n+nn}{stats}
\end{Verbatim}


    \begin{Verbatim}[commandchars=\\\{\}]
{\color{incolor}In [{\color{incolor}2}]:} \PY{k}{def} \PY{n+nf}{form\PYZus{}genome}\PY{p}{(}\PY{n}{T}\PY{p}{,} \PY{n}{d}\PY{p}{)}\PY{p}{:}
            \PY{n}{res}  \PY{o}{=} \PY{p}{[}\PY{l+m+mi}{0}\PY{p}{]}\PY{o}{*}\PY{n}{T}
            \PY{k}{for} \PY{n}{i} \PY{o+ow}{in} \PY{n+nb}{range}\PY{p}{(}\PY{n+nb}{len}\PY{p}{(}\PY{n}{res}\PY{p}{)}\PY{p}{)}\PY{p}{:}
                \PY{n}{rand} \PY{o}{=} \PY{n}{np}\PY{o}{.}\PY{n}{random}\PY{o}{.}\PY{n}{randn}\PY{p}{(}\PY{p}{)}
                \PY{k}{if} \PY{n}{rand} \PY{o}{\PYZlt{}} \PY{l+m+mi}{0}\PY{p}{:}
                    \PY{n}{res}\PY{p}{[}\PY{n}{i}\PY{p}{]} \PY{o}{+}\PY{o}{=} \PY{n}{d}\PY{p}{[}\PY{l+m+mi}{0}\PY{p}{]}
                \PY{k}{if} \PY{n}{rand} \PY{o}{\PYZgt{}}\PY{o}{=} \PY{l+m+mi}{0}\PY{p}{:}
                    \PY{n}{res}\PY{p}{[}\PY{n}{i}\PY{p}{]} \PY{o}{+}\PY{o}{=} \PY{n}{d}\PY{p}{[}\PY{l+m+mi}{1}\PY{p}{]}
            \PY{k}{return} \PY{n}{res}
        
        
        \PY{n}{form\PYZus{}genome}\PY{p}{(}\PY{l+m+mi}{10}\PY{p}{,} \PY{p}{[}\PY{o}{\PYZhy{}}\PY{l+m+mi}{1}\PY{p}{,}\PY{l+m+mi}{1}\PY{p}{]}\PY{p}{)}
\end{Verbatim}


\begin{Verbatim}[commandchars=\\\{\}]
{\color{outcolor}Out[{\color{outcolor}2}]:} [1, -1, -1, 1, 1, 1, 1, -1, -1, 1]
\end{Verbatim}
            
    \emph{Écrire la fonction qui calcule le coût d'un génome. Elle prend en
paramètre d'entrée : un génome et \(r_0\) , la borne de l'intervalle.
Elle retourne un scalaire.}

    \begin{Verbatim}[commandchars=\\\{\}]
{\color{incolor}In [{\color{incolor}71}]:} \PY{k}{def} \PY{n+nf}{cost}\PY{p}{(}\PY{n}{genome}\PY{p}{,} \PY{n}{r0}\PY{p}{)}\PY{p}{:}
             \PY{n}{price} \PY{o}{=} \PY{l+m+mi}{0}
             \PY{k}{for} \PY{n}{i} \PY{o+ow}{in} \PY{n+nb}{range}\PY{p}{(}\PY{l+m+mi}{1}\PY{p}{,}\PY{n+nb}{len}\PY{p}{(}\PY{n}{genome}\PY{p}{)}\PY{o}{+}\PY{l+m+mi}{1}\PY{p}{)}\PY{p}{:}
                 \PY{k}{if} \PY{n+nb}{sum}\PY{p}{(}\PY{n}{genome}\PY{p}{[}\PY{p}{:}\PY{n}{i}\PY{p}{]}\PY{p}{)} \PY{o}{\PYZlt{}} \PY{l+m+mi}{0}\PY{p}{:}
                     \PY{k}{if} \PY{n+nb}{sum}\PY{p}{(}\PY{n}{genome}\PY{p}{[}\PY{p}{:}\PY{n}{i}\PY{p}{]}\PY{p}{)} \PY{o}{\PYZlt{}}\PY{o}{=} \PY{o}{\PYZhy{}}\PY{n}{r0}\PY{p}{:}
                         \PY{n}{price} \PY{o}{+}\PY{o}{=} \PY{l+m+mi}{1} \PY{c+c1}{\PYZsh{} \PYZhy{}sum(genome[:i]) \PYZhy{} r0}
                 \PY{k}{else} \PY{p}{:}
                     \PY{k}{if} \PY{n+nb}{sum}\PY{p}{(}\PY{n}{genome}\PY{p}{[}\PY{p}{:}\PY{n}{i}\PY{p}{]}\PY{p}{)} \PY{o}{\PYZgt{}}\PY{o}{=} \PY{n}{r0}\PY{p}{:}
                         \PY{n}{price} \PY{o}{+}\PY{o}{=} \PY{l+m+mi}{1} \PY{c+c1}{\PYZsh{} sum(genome[:i]) \PYZhy{} r0}
             \PY{k}{return} \PY{n}{price}
         
         \PY{n}{gen1} \PY{o}{=} \PY{n}{form\PYZus{}genome}\PY{p}{(}\PY{l+m+mi}{5}\PY{p}{,} \PY{p}{[}\PY{o}{\PYZhy{}}\PY{l+m+mi}{1}\PY{p}{,}\PY{l+m+mi}{1}\PY{p}{]}\PY{p}{)}
         \PY{n+nb}{print}\PY{p}{(}\PY{n}{gen1}\PY{p}{)}
         \PY{n}{cost}\PY{p}{(}\PY{n}{gen1}\PY{p}{,} \PY{l+m+mi}{2}\PY{p}{)}
\end{Verbatim}


    \begin{Verbatim}[commandchars=\\\{\}]
[1, -1, -1, 1, -1]

    \end{Verbatim}

\begin{Verbatim}[commandchars=\\\{\}]
{\color{outcolor}Out[{\color{outcolor}71}]:} 0
\end{Verbatim}
            
    \subsection{Codage de l'algorithme
génétique}\label{codage-de-lalgorithme-guxe9nuxe9tique}

    \emph{Initialisation : écrire une fonction qui créée aléatoirement une
population initiale de N génomes de taille \(T\).}

    \begin{Verbatim}[commandchars=\\\{\}]
{\color{incolor}In [{\color{incolor}4}]:} \PY{k}{def} \PY{n+nf}{get\PYZus{}population}\PY{p}{(}\PY{n}{N}\PY{p}{,} \PY{n}{T}\PY{p}{)}\PY{p}{:}
            \PY{n}{population} \PY{o}{=} \PY{p}{[}\PY{p}{]}
            \PY{k}{for} \PY{n}{i} \PY{o+ow}{in} \PY{n+nb}{range}\PY{p}{(}\PY{n}{N}\PY{p}{)}\PY{p}{:}
                \PY{n}{population}\PY{o}{.}\PY{n}{append}\PY{p}{(}\PY{n}{form\PYZus{}genome}\PY{p}{(}\PY{n}{T}\PY{p}{,} \PY{p}{[}\PY{o}{\PYZhy{}}\PY{l+m+mi}{1}\PY{p}{,}\PY{l+m+mi}{1}\PY{p}{]}\PY{p}{)}\PY{p}{)}
            \PY{k}{return} \PY{n}{population}
        
        \PY{n}{get\PYZus{}population}\PY{p}{(}\PY{l+m+mi}{10}\PY{p}{,} \PY{l+m+mi}{10}\PY{p}{)}
\end{Verbatim}


\begin{Verbatim}[commandchars=\\\{\}]
{\color{outcolor}Out[{\color{outcolor}4}]:} [[1, 1, 1, -1, -1, 1, -1, -1, 1, -1],
         [-1, -1, -1, 1, -1, -1, 1, -1, -1, -1],
         [-1, -1, 1, -1, 1, 1, 1, -1, 1, 1],
         [-1, -1, 1, -1, 1, 1, -1, 1, 1, -1],
         [1, -1, -1, 1, -1, -1, -1, 1, 1, 1],
         [1, 1, -1, 1, 1, -1, 1, -1, 1, -1],
         [1, -1, 1, -1, -1, 1, 1, -1, 1, -1],
         [1, -1, 1, 1, -1, -1, -1, -1, 1, -1],
         [1, -1, -1, -1, 1, -1, 1, -1, 1, -1],
         [1, -1, 1, -1, -1, 1, 1, 1, 1, -1]]
\end{Verbatim}
            
    \emph{Écrire une fonction qui trie les génomes d'une population en
fonction de leur coût. Indice : utiliser la fonction numpy.argsort().}

    \begin{Verbatim}[commandchars=\\\{\}]
{\color{incolor}In [{\color{incolor}5}]:} \PY{k}{def} \PY{n+nf}{sort\PYZus{}pop}\PY{p}{(}\PY{n}{pop}\PY{p}{,} \PY{n}{r0}\PY{p}{)}\PY{p}{:}
            \PY{n}{cost\PYZus{}array} \PY{o}{=} \PY{p}{[}\PY{p}{]}
            \PY{k}{for} \PY{n}{i} \PY{o+ow}{in} \PY{n}{pop}\PY{p}{:}
                \PY{n}{cost\PYZus{}array}\PY{o}{.}\PY{n}{append}\PY{p}{(}\PY{n}{cost}\PY{p}{(}\PY{n}{i}\PY{p}{,} \PY{n}{r0}\PY{p}{)}\PY{p}{)}
            \PY{c+c1}{\PYZsh{} print(cost\PYZus{}array)}
            \PY{n}{indexs} \PY{o}{=} \PY{n}{np}\PY{o}{.}\PY{n}{argsort}\PY{p}{(}\PY{n}{cost\PYZus{}array}\PY{p}{)}
            \PY{c+c1}{\PYZsh{} print(indexs)}
            \PY{n}{sorted\PYZus{}pop} \PY{o}{=} \PY{p}{[}\PY{p}{]}
            \PY{k}{for} \PY{n}{i} \PY{o+ow}{in} \PY{n+nb}{range}\PY{p}{(}\PY{n+nb}{len}\PY{p}{(}\PY{n}{indexs}\PY{p}{)}\PY{p}{)}\PY{p}{:}
                \PY{n}{sorted\PYZus{}pop}\PY{o}{.}\PY{n}{append}\PY{p}{(}\PY{n}{pop}\PY{p}{[}\PY{n}{indexs}\PY{p}{[}\PY{n}{i}\PY{p}{]}\PY{p}{]}\PY{p}{)}
            \PY{k}{return} \PY{n}{sorted\PYZus{}pop}
        
        
        \PY{n}{my\PYZus{}pop} \PY{o}{=} \PY{n}{get\PYZus{}population}\PY{p}{(}\PY{l+m+mi}{5}\PY{p}{,} \PY{l+m+mi}{5}\PY{p}{)}
        \PY{n+nb}{print}\PY{p}{(}\PY{n}{my\PYZus{}pop}\PY{p}{)}
        \PY{n}{sort\PYZus{}pop}\PY{p}{(}\PY{n}{my\PYZus{}pop}\PY{p}{,} \PY{l+m+mi}{2}\PY{p}{)}
\end{Verbatim}


    \begin{Verbatim}[commandchars=\\\{\}]
[[1, 1, 1, 1, 1], [1, -1, -1, -1, 1], [-1, -1, -1, 1, -1], [1, 1, 1, -1, 1], [1, -1, -1, -1, 1]]

    \end{Verbatim}

\begin{Verbatim}[commandchars=\\\{\}]
{\color{outcolor}Out[{\color{outcolor}5}]:} [[1, -1, -1, -1, 1],
         [1, -1, -1, -1, 1],
         [-1, -1, -1, 1, -1],
         [1, 1, 1, -1, 1],
         [1, 1, 1, 1, 1]]
\end{Verbatim}
            
    \emph{Écrire une fonction qui sélectionne les \(N_s\) génomes ayant les
valeurs de coût les plus importantes (sélection par rang). On prendra
par exemple \(N_s = N/2\). La valeur de sortie correspondra à la
population à faire muter et/ou à croiser.}

    \begin{Verbatim}[commandchars=\\\{\}]
{\color{incolor}In [{\color{incolor}6}]:} \PY{k}{def} \PY{n+nf}{selection}\PY{p}{(}\PY{n}{pop}\PY{p}{,} \PY{n}{r0}\PY{p}{)}\PY{p}{:}
            \PY{k}{return} \PY{n}{sort\PYZus{}pop}\PY{p}{(}\PY{n}{pop}\PY{p}{,} \PY{n}{r0}\PY{p}{)}\PY{p}{[}\PY{p}{(}\PY{n+nb}{len}\PY{p}{(}\PY{n}{pop}\PY{p}{)}\PY{o}{/}\PY{o}{/}\PY{l+m+mi}{2}\PY{p}{)}\PY{p}{:}\PY{p}{]}
        
        \PY{n}{selection}\PY{p}{(}\PY{n}{my\PYZus{}pop}\PY{p}{,} \PY{l+m+mi}{2}\PY{p}{)}
\end{Verbatim}


\begin{Verbatim}[commandchars=\\\{\}]
{\color{outcolor}Out[{\color{outcolor}6}]:} [[-1, -1, -1, 1, -1], [1, 1, 1, -1, 1], [1, 1, 1, 1, 1]]
\end{Verbatim}
            
    \emph{Nous allons maintenant procéder à la mutation des génomes
sélectionnés. La mutation est définie par un taux \(T_m\) qui correspond
à la probabilité qu'un gène \(a\) de muter. Il s'agit donc ici de
parcourir les gènes de chaque génome et de les faire muter avec une
probabilité \(T_m\). Écrire la fonction qui permet d'effectuer cette
mutation sur la population sélectionnée.}

    \begin{Verbatim}[commandchars=\\\{\}]
{\color{incolor}In [{\color{incolor}7}]:} \PY{k}{def} \PY{n+nf}{mut\PYZus{}genome}\PY{p}{(}\PY{n}{genome}\PY{p}{,} \PY{n}{Tm}\PY{p}{)}\PY{p}{:}
            \PY{n}{new\PYZus{}genome} \PY{o}{=} \PY{p}{[}\PY{p}{]}
            \PY{k}{for} \PY{n}{i} \PY{o+ow}{in} \PY{n}{genome}\PY{p}{:}
                \PY{n}{rand} \PY{o}{=} \PY{n}{np}\PY{o}{.}\PY{n}{random}\PY{o}{.}\PY{n}{randn}\PY{p}{(}\PY{p}{)}
                \PY{k}{if} \PY{n+nb}{abs}\PY{p}{(}\PY{n}{rand}\PY{p}{)} \PY{o}{\PYZlt{}} \PY{n}{Tm}\PY{p}{:}
                    \PY{n}{new\PYZus{}genome}\PY{o}{.}\PY{n}{append}\PY{p}{(}\PY{o}{\PYZhy{}}\PY{n}{i}\PY{p}{)}
                \PY{k}{else} \PY{p}{:} 
                    \PY{n}{new\PYZus{}genome}\PY{o}{.}\PY{n}{append}\PY{p}{(}\PY{n}{i}\PY{p}{)}
            \PY{k}{return} \PY{n}{new\PYZus{}genome}
                    
        \PY{k}{def} \PY{n+nf}{mut\PYZus{}pop}\PY{p}{(}\PY{n}{pop}\PY{p}{,} \PY{n}{Tm}\PY{p}{)}\PY{p}{:}
            \PY{n}{new\PYZus{}pop} \PY{o}{=} \PY{p}{[}\PY{p}{]}
            \PY{k}{for} \PY{n}{i} \PY{o+ow}{in} \PY{n}{pop}\PY{p}{:}
                \PY{n}{new\PYZus{}pop}\PY{o}{.}\PY{n}{append}\PY{p}{(}\PY{n}{mut\PYZus{}genome}\PY{p}{(}\PY{n}{i}\PY{p}{,} \PY{n}{Tm}\PY{p}{)}\PY{p}{)}
            \PY{k}{return} \PY{n}{new\PYZus{}pop}
        
        \PY{n+nb}{print}\PY{p}{(}\PY{n}{selection}\PY{p}{(}\PY{n}{my\PYZus{}pop}\PY{p}{,} \PY{l+m+mi}{2}\PY{p}{)}\PY{p}{)}
        \PY{n}{mut\PYZus{}pop}\PY{p}{(}\PY{n}{selection}\PY{p}{(}\PY{n}{my\PYZus{}pop}\PY{p}{,} \PY{l+m+mi}{2}\PY{p}{)}\PY{p}{,} \PY{l+m+mf}{0.5}\PY{p}{)}
                    
                    
\end{Verbatim}


    \begin{Verbatim}[commandchars=\\\{\}]
[[-1, -1, -1, 1, -1], [1, 1, 1, -1, 1], [1, 1, 1, 1, 1]]

    \end{Verbatim}

\begin{Verbatim}[commandchars=\\\{\}]
{\color{outcolor}Out[{\color{outcolor}7}]:} [[-1, -1, 1, 1, -1], [1, 1, -1, 1, 1], [1, 1, 1, 1, -1]]
\end{Verbatim}
            
    \emph{Nous allons maintenant effectuer le croisement de la population
mutée. Le croisement se définit par un taux \(T_c\) qui indique la
probabilité avec laquelle un croisement entre génomes peut intervenir.
Coder la fonction qui effectue le croisement entre individus de la
population sélectionnée avec la probabilité \(T_c\). La position du
croisement dans le génome, ainsi que l'individu avec lequel le
croisement sera effectué seront tirés aléatoirement.}

    \begin{Verbatim}[commandchars=\\\{\}]
{\color{incolor}In [{\color{incolor}63}]:} \PY{k}{def} \PY{n+nf}{cross\PYZus{}over}\PY{p}{(}\PY{n}{pop}\PY{p}{,} \PY{n}{Tc}\PY{p}{)}\PY{p}{:}
         
             \PY{k}{for} \PY{n}{i} \PY{o+ow}{in} \PY{n+nb}{range}\PY{p}{(}\PY{n+nb}{len}\PY{p}{(}\PY{n}{pop}\PY{p}{)}\PY{p}{)}\PY{p}{:}
                 \PY{n}{g} \PY{o}{=} \PY{n}{pop}\PY{p}{[}\PY{n}{i}\PY{p}{]}
                 \PY{k}{if} \PY{n}{random}\PY{o}{.}\PY{n}{random}\PY{p}{(}\PY{p}{)} \PY{o}{\PYZgt{}} \PY{n}{Tc}\PY{p}{:}
                     \PY{n}{pos} \PY{o}{=} \PY{n+nb}{int}\PY{p}{(}\PY{n}{random}\PY{o}{.}\PY{n}{random}\PY{p}{(}\PY{p}{)}\PY{o}{*}\PY{n+nb}{len}\PY{p}{(}\PY{n}{g}\PY{p}{)}\PY{p}{)}
                     \PY{n}{sec\PYZus{}g} \PY{o}{=} \PY{p}{[}\PY{p}{]}
                     \PY{n}{r} \PY{o}{=} \PY{l+m+mi}{0}
                     \PY{k}{while} \PY{n}{sec\PYZus{}g} \PY{o}{==} \PY{p}{[}\PY{p}{]} \PY{o+ow}{or} \PY{n}{sec\PYZus{}g} \PY{o}{==} \PY{n}{g}\PY{p}{:}
                         \PY{n}{r} \PY{o}{=} \PY{n+nb}{int}\PY{p}{(}\PY{n}{random}\PY{o}{.}\PY{n}{random}\PY{p}{(}\PY{p}{)}\PY{o}{*}\PY{n+nb}{len}\PY{p}{(}\PY{n}{pop}\PY{p}{)}\PY{p}{)}
                         \PY{n}{sec\PYZus{}g} \PY{o}{=} \PY{n}{pop}\PY{p}{[}\PY{n}{r}\PY{p}{]}
                     \PY{c+c1}{\PYZsh{}print(\PYZsq{}Crossing\PYZhy{}over à la position \PYZsq{} + str(pos) +}
                      \PY{c+c1}{\PYZsh{}    \PYZsq{} de \PYZsq{} + str(g) + \PYZsq{} et \PYZsq{} + str(sec\PYZus{}g))}
                     \PY{n}{pop}\PY{p}{[}\PY{n}{i}\PY{p}{]} \PY{o}{=} \PY{n}{g}\PY{p}{[}\PY{p}{:}\PY{n}{pos}\PY{p}{]} \PY{o}{+} \PY{n}{sec\PYZus{}g}\PY{p}{[}\PY{n}{pos}\PY{p}{:}\PY{p}{]}
                     \PY{n}{pop}\PY{p}{[}\PY{n}{r}\PY{p}{]} \PY{o}{=} \PY{n}{sec\PYZus{}g}\PY{p}{[}\PY{p}{:}\PY{n}{pos}\PY{p}{]} \PY{o}{+} \PY{n}{g}\PY{p}{[}\PY{n}{pos}\PY{p}{:}\PY{p}{]}
                     
         
         \PY{n}{my\PYZus{}sel} \PY{o}{=} \PY{n}{selection}\PY{p}{(}\PY{n}{my\PYZus{}pop}\PY{p}{,} \PY{l+m+mi}{2}\PY{p}{)}
         \PY{n+nb}{print}\PY{p}{(}\PY{n}{my\PYZus{}sel}\PY{p}{)}
         \PY{n}{cross\PYZus{}over}\PY{p}{(}\PY{n}{my\PYZus{}sel}\PY{p}{,} \PY{l+m+mf}{0.5}\PY{p}{)}
         \PY{n+nb}{print}\PY{p}{(}\PY{n}{my\PYZus{}sel}\PY{p}{)}
\end{Verbatim}


    \begin{Verbatim}[commandchars=\\\{\}]
[[-1, -1, -1, 1, -1], [1, 1, 1, -1, 1], [1, 1, 1, 1, 1]]
[[-1, -1, -1, -1, 1], [1, 1, 1, 1, -1], [1, 1, 1, 1, 1]]

    \end{Verbatim}

    \emph{Mettre à jour la population courante avec la population mutée et
incrémenter le compteur de génération.}

    \begin{Verbatim}[commandchars=\\\{\}]
{\color{incolor}In [{\color{incolor}64}]:} \PY{k}{def} \PY{n+nf}{evolve}\PY{p}{(}\PY{n}{N}\PY{p}{,} \PY{n}{T}\PY{p}{,} \PY{n}{r0}\PY{p}{,} \PY{n}{Tm}\PY{p}{,} \PY{n}{Tc}\PY{p}{,} \PY{n}{it}\PY{p}{)}\PY{p}{:}
             \PY{n}{pop} \PY{o}{=} \PY{n}{get\PYZus{}population}\PY{p}{(}\PY{n}{N}\PY{p}{,} \PY{n}{T}\PY{p}{)}
             \PY{n}{costs\PYZus{}gener} \PY{o}{=} \PY{p}{[}\PY{p}{]}
             \PY{k}{for} \PY{n}{i} \PY{o+ow}{in} \PY{n+nb}{range}\PY{p}{(}\PY{n}{it}\PY{p}{)}\PY{p}{:}
                 \PY{n+nb}{print}\PY{p}{(}\PY{l+s+s1}{\PYZsq{}}\PY{l+s+s1}{Generation }\PY{l+s+s1}{\PYZsq{}} \PY{o}{+} \PY{n+nb}{str}\PY{p}{(}\PY{n}{i}\PY{p}{)}\PY{p}{)}
                 \PY{c+c1}{\PYZsh{}print(pop)}
                 \PY{n}{sel} \PY{o}{=} \PY{n}{selection}\PY{p}{(}\PY{n}{pop}\PY{p}{,} \PY{n}{r0}\PY{p}{)}
                 \PY{c+c1}{\PYZsh{}print(len(sel))}
                 \PY{n}{changed\PYZus{}sel} \PY{o}{=} \PY{n}{mut\PYZus{}pop}\PY{p}{(}\PY{n}{sel}\PY{p}{,} \PY{n}{Tm}\PY{p}{)}
                 \PY{n}{cross\PYZus{}over}\PY{p}{(}\PY{n}{changed\PYZus{}sel}\PY{p}{,} \PY{n}{Tc}\PY{p}{)}
                 \PY{n}{pop} \PY{o}{=} \PY{n}{changed\PYZus{}sel} \PY{o}{+} \PY{n}{sort\PYZus{}pop}\PY{p}{(}\PY{n}{pop}\PY{p}{,} \PY{n}{r0}\PY{p}{)}\PY{p}{[}\PY{p}{:}\PY{p}{(}\PY{n+nb}{len}\PY{p}{(}\PY{n}{pop}\PY{p}{)}\PY{o}{/}\PY{o}{/}\PY{l+m+mi}{2}\PY{p}{)}\PY{p}{]}
                 \PY{n}{costs\PYZus{}gener}\PY{o}{.}\PY{n}{append}\PY{p}{(}\PY{n}{costs\PYZus{}pop}\PY{p}{(}\PY{n}{pop}\PY{p}{,} \PY{n}{r0}\PY{p}{)}\PY{p}{[}\PY{l+m+mi}{0}\PY{p}{]}\PY{p}{)}
             \PY{c+c1}{\PYZsh{}print(range(it), costs\PYZus{}gener)}
             \PY{n}{plt}\PY{o}{.}\PY{n}{plot}\PY{p}{(}\PY{n+nb}{range}\PY{p}{(}\PY{n}{it}\PY{p}{)}\PY{p}{,} \PY{n}{costs\PYZus{}gener}\PY{p}{)}
             \PY{n}{plt}\PY{o}{.}\PY{n}{show}\PY{p}{(}\PY{p}{)}
                 
             \PY{k}{return} \PY{n}{pop}
                 
         \PY{n}{evolve}\PY{p}{(}\PY{l+m+mi}{5}\PY{p}{,} \PY{l+m+mi}{5}\PY{p}{,} \PY{l+m+mi}{2}\PY{p}{,} \PY{l+m+mf}{0.5}\PY{p}{,} \PY{l+m+mf}{0.5}\PY{p}{,} \PY{l+m+mi}{10}\PY{p}{)}
             
\end{Verbatim}


    \begin{Verbatim}[commandchars=\\\{\}]
Generation 0
Generation 1
Generation 2
Generation 3
Generation 4
Generation 5
Generation 6
Generation 7
Generation 8
Generation 9

    \end{Verbatim}

    \begin{center}
    \adjustimage{max size={0.9\linewidth}{0.9\paperheight}}{output_21_1.png}
    \end{center}
    { \hspace*{\fill} \\}
    
\begin{Verbatim}[commandchars=\\\{\}]
{\color{outcolor}Out[{\color{outcolor}64}]:} [[-1, 1, -1, 1, -1],
          [1, 1, 1, 1, -1],
          [1, -1, 1, 1, -1],
          [-1, 1, 1, -1, -1],
          [1, 1, -1, 1, -1]]
\end{Verbatim}
            
    \emph{Calculer le coût moyen de la nouvelle population, ainsi que le
coût minimum obtenu.}

    \begin{Verbatim}[commandchars=\\\{\}]
{\color{incolor}In [{\color{incolor}65}]:} \PY{k}{def} \PY{n+nf}{costs\PYZus{}pop}\PY{p}{(}\PY{n}{pop}\PY{p}{,} \PY{n}{r0}\PY{p}{)}\PY{p}{:}
             \PY{n}{costs} \PY{o}{=} \PY{p}{[}\PY{p}{]}
             \PY{k}{for} \PY{n}{i} \PY{o+ow}{in} \PY{n+nb}{range}\PY{p}{(}\PY{n+nb}{len}\PY{p}{(}\PY{n}{pop}\PY{p}{)}\PY{p}{)}\PY{p}{:}
                 \PY{n}{costs}\PY{o}{.}\PY{n}{append}\PY{p}{(}\PY{n}{cost}\PY{p}{(}\PY{n}{pop}\PY{p}{[}\PY{n}{i}\PY{p}{]}\PY{p}{,} \PY{n}{r0}\PY{p}{)}\PY{p}{)}
             \PY{k}{return} \PY{p}{(}\PY{n}{stats}\PY{o}{.}\PY{n}{mean}\PY{p}{(}\PY{n}{costs}\PY{p}{)}\PY{p}{,} \PY{n+nb}{min}\PY{p}{(}\PY{n}{costs}\PY{p}{)}\PY{p}{,} \PY{n+nb}{max}\PY{p}{(}\PY{n}{costs}\PY{p}{)}\PY{p}{)}
         
         \PY{n}{costs\PYZus{}pop}\PY{p}{(}\PY{n}{my\PYZus{}pop}\PY{p}{,} \PY{l+m+mi}{2}\PY{p}{)}
\end{Verbatim}


\begin{Verbatim}[commandchars=\\\{\}]
{\color{outcolor}Out[{\color{outcolor}65}]:} (2, 0, 6)
\end{Verbatim}
            
    \emph{Quel(s) critère(s) d'arrêt utiliseriez-vous ? Faire tourner
l'algorithme avec \(T = 500\), \(r_0 = 4\), \(N = 100\), \(T_m = 0.05\)
et \(T_c = 0.1\). Tracer l'évolution du coût moyen au fur et à mesure
des générations.}

    \begin{Verbatim}[commandchars=\\\{\}]
{\color{incolor}In [{\color{incolor}74}]:} \PY{n}{evolve}\PY{p}{(}\PY{l+m+mi}{100}\PY{p}{,} \PY{l+m+mi}{500}\PY{p}{,} \PY{l+m+mi}{4}\PY{p}{,} \PY{l+m+mf}{0.05}\PY{p}{,} \PY{l+m+mf}{0.1}\PY{p}{,} \PY{l+m+mi}{100}\PY{p}{)}
         \PY{l+m+mi}{1}
\end{Verbatim}


    \begin{Verbatim}[commandchars=\\\{\}]
Generation 0
Generation 1
Generation 2
Generation 3
Generation 4
Generation 5
Generation 6
Generation 7
Generation 8
Generation 9
Generation 10
Generation 11
Generation 12
Generation 13
Generation 14
Generation 15
Generation 16
Generation 17
Generation 18
Generation 19
Generation 20
Generation 21
Generation 22
Generation 23
Generation 24
Generation 25
Generation 26
Generation 27
Generation 28
Generation 29
Generation 30
Generation 31
Generation 32
Generation 33
Generation 34
Generation 35
Generation 36
Generation 37
Generation 38
Generation 39
Generation 40
Generation 41
Generation 42
Generation 43
Generation 44
Generation 45
Generation 46
Generation 47
Generation 48
Generation 49
Generation 50
Generation 51
Generation 52
Generation 53
Generation 54
Generation 55
Generation 56
Generation 57
Generation 58
Generation 59
Generation 60
Generation 61
Generation 62
Generation 63
Generation 64
Generation 65
Generation 66
Generation 67
Generation 68
Generation 69
Generation 70
Generation 71
Generation 72
Generation 73
Generation 74
Generation 75
Generation 76
Generation 77
Generation 78
Generation 79
Generation 80
Generation 81
Generation 82
Generation 83
Generation 84
Generation 85
Generation 86
Generation 87
Generation 88
Generation 89
Generation 90
Generation 91
Generation 92
Generation 93
Generation 94
Generation 95
Generation 96
Generation 97
Generation 98
Generation 99

    \end{Verbatim}

    \begin{center}
    \adjustimage{max size={0.9\linewidth}{0.9\paperheight}}{output_25_1.png}
    \end{center}
    { \hspace*{\fill} \\}
    
\begin{Verbatim}[commandchars=\\\{\}]
{\color{outcolor}Out[{\color{outcolor}74}]:} 1
\end{Verbatim}
            
    \emph{Faire varier les paramètres suivants (\(N\), \(r_0\) , \(T_m\) et
\(T_c\)) pour étudier leur influence.}

    \emph{Décrire le comportement de l'algorithme (évolution des coûts moyen
et minimum) et interpréter la convergence en fonction des paramètres
utilisés.}

    \emph{Quelles différences remarquez-vous par rapport aux méthodes
différentielles étudiées jusqu'à maintenant (descente de gradient,
Newton) ?}


    % Add a bibliography block to the postdoc
    
    
    
    \end{document}
